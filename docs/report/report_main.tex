\documentclass[12pt,a4paper]{article}

% Pacotes básicos
\usepackage[utf8]{inputenc}
\usepackage[T1]{fontenc}
\usepackage[brazil]{babel}
\usepackage{graphicx}
\usepackage{float}
\usepackage{amsmath, amssymb}
\usepackage{hyperref}
\usepackage{caption}
\usepackage{cite}
\usepackage{listings}
\usepackage{enumitem}
\usepackage{xcolor}
\usepackage{subcaption} 
\usepackage{booktabs} % Para tabelas mais bonitas

% Configurações do listings (Python)
\lstset{
  language=Python,
  basicstyle=\ttfamily\footnotesize,
  numbers=left,
  numberstyle=\tiny,
  frame=single,
  breaklines=true,
  keywordstyle=\color{blue}\bfseries,
  stringstyle=\color{red},
  commentstyle=\color{green!60!black}\itshape,
  showstringspaces=false,
  inputencoding=utf8,
  extendedchars=true,
  literate={á}{{\'a}}1 {ã}{{\~a}}1 {é}{{\'e}}1 {ç}{{\c{c}}}1 {ê}{{\^e}}1 {ó}{{\'o}}1 {í}{{\'i}}1
}

\graphicspath{{./}{../}{../../}{../../plots/}{../../plots/img/}}

\begin{document}

% ==============================
% CAPA
% ==============================
\begin{titlepage}
    \centering
    {\Large \textbf{Universidade Federal de Minas Gerais}}\\[0.3cm]
    {\large Engenharia de Sistemas}\\[2cm]
    
    {\Huge \textbf{Relatório de controle de um robô em um ambiente simulado usando aprendizado por reforço (Q-Learning)}}\\[1.5cm]
    
    \textbf{Fundamentos de Inteligência Artificial}\\[0.5cm]
    \textbf{Professores:} Cristiano Castro e João Paulo Lara\\[1.5cm]
    
    \begin{flushleft}
        \textbf{Alunos:}\\
        Áquila Oliveira Souza --- 2021019327\\
        Arthur Jorge --- 2022055718\\
        Felippe Veloso Marinho --- 2021072260\\
        Jefferson Pereira de Souza --- 2022099049\\
        Josoé Santos Queiroz --- 2019026982
    \end{flushleft}
    
    \vfill
    {\large Belo Horizonte, MG}\\
    {\large \today}
\end{titlepage}

\clearpage
\tableofcontents
\clearpage

% ==============================
% 1. INTRODUÇÃO
% ==============================
\section{Introdução}
O objetivo deste trabalho é documentar a implementação do algoritmo Q-learning seguindo uma política $\epsilon$-greedy para ensinar um agente a navegar em um laboratório simulado e encontrar a saída com o mínimo de passos possível, evitando obstáculos que possam atolá-lo ou destruí-lo.

% ==============================
% 2. FUNDAMENTAÇÃO TEÓRICA (NOVO)
% ==============================
\section{Fundamentação Teórica: Q-Learning}

O Q-Learning é um algoritmo de Aprendizado por Reforço \textit{model-free} (livre de modelo) e \textit{off-policy}. O objetivo do algoritmo é aprender uma função de valor de ação $Q(s, a)$, que estima a recompensa acumulada esperada ao executar uma ação $a$ em um estado $s$ e, posteriormente, seguir uma política ótima.

A base do algoritmo é a Equação de Bellman para a atualização iterativa dos valores Q. A regra de atualização utilizada a cada passo de tempo é dada por:

\begin{equation}
    Q(s,a) \leftarrow Q(s,a) + \alpha \left[ R + \gamma \max_{a'} Q(s', a') - Q(s,a) \right]
\end{equation}

Onde:
\begin{itemize}
    \item $Q(s,a)$: Valor atual estimado para o par estado-ação.
    \item $\alpha$ (taxa de aprendizado): Determina o quanto as novas informações substituem as antigas ($0 < \alpha \le 1$).
    \item $R$: Recompensa imediata recebida após a ação.
    \item $\gamma$ (fator de desconto): Determina a importância das recompensas futuras ($0 \le \gamma \le 1$).
    \item $\max_{a'} Q(s', a')$: A estimativa da melhor recompensa futura possível a partir do novo estado $s'$.
\end{itemize}

O algoritmo garante a convergência para os valores ótimos $Q^*(s,a)$ desde que todos os pares estado-ação sejam visitados infinitas vezes e a taxa de aprendizado decaia apropriadamente, permitindo que o agente derive uma política ótima $\pi^*(s) = \arg\max_a Q(s,a)$.

% ==============================
% 3. MODELAGEM DO PROBLEMA
% ==============================
\section{Modelagem do Problema}

\subsection{Definição do Ambiente e Recompensas}

O espaço de estados é um grid $4 \times 4$. As ações do agente são: ir para CIMA, DIREITA, BAIXO e ESQUERDA, desde que os limites do grid permitam. Por exemplo, na posição $(1, 1)$, ele só pode ir para cima ou para a direita.

A função de recompensa é definida da seguinte forma:
\begin{itemize}
    \item Cada passo para um estado vazio gera uma recompensa de $-1$.
    \item Se pisar na lama, a recompensa é $-5$.
    \item Se pisar na substância tóxica, a recompensa é $-20$ e o episódio termina (estado terminal negativo).
    \item Se encontrar a saída, a recompensa é $+20$ e o episódio termina (estado terminal positivo).
\end{itemize}

Os estados terminais representam os locais onde as substâncias tóxicas se encontram e também a porta de saída do laboratório, conforme ilustrado na Figura \ref{fig:grid}.

\begin{figure}[H]
    \centering
    \includegraphics[width=0.4\textwidth]{grid.png}
    \caption{Representação do ambiente do laboratório (Grid $4 \times 4$).}
    \label{fig:grid}
\end{figure}

% ==============================
% 4. ESTRATÉGIAS ADOTADAS
% ==============================
\section{Estratégias Adotadas}

\subsection{Implementação Computacional}
As características do ambiente foram traduzidas para o código da seguinte maneira:

\begin{itemize}
    \item \textbf{Espaço de estados:} Formado por 16 células, indexadas por coordenadas $(x, y)$ onde $x, y \in \{1, 2, 3, 4\}$.
    \item \textbf{Espaço de ações:} Definido pela lista \texttt{actions = ["CIMA", "DIREITA", "BAIXO", "ESQUERDA"]}, mapeados respectivamente para os índices 0, 1, 2 e 3.
    \item \textbf{Dinâmica:} Foi definida a função \texttt{rollout} que move o agente uma célula na direção escolhida. Não há verificação de borda na dinâmica de movimento em si dentro desta função; no entanto, as ações inválidas são prevenidas marcando-as com $-\infty$ na tabela Q.
    \item \textbf{Recompensas (\texttt{get\_reward}):}
    \begin{itemize}
        \item \textit{Estados terminais negativos (Tóxicos/Radioativos):} As posições $(4,2)$ e $(1,3)$ retornam $-20$.
        \item \textit{Obstáculos (Lama):} As posições $(3,2)$ e $(2,4)$ retornam $-5$.
        \item \textit{Estado terminal positivo (Saída):} A posição $(4,4)$ retorna $+20$.
        \item \textit{Passo comum:} Retorna $-1$ para os demais estados.
    \end{itemize}
    \item \textbf{Condições de parada:} Os episódios encerram ao atingir uma recompensa de $+20$ ou $-20$, ou ao ultrapassar o limite de passos (\texttt{limit = 10}).
\end{itemize}

\subsection{Hiperparâmetros}
A tabela Q é inicializada com zeros para ações válidas e $-\infty$ para ações inválidas (bordas). Os hiperparâmetros utilizados foram:
\begin{itemize}
    \item Taxa de aprendizado ($\alpha$): $0.2$
    \item Fator de desconto ($\gamma$): $0.95$
    \item Máximo de iterações por episódio: $10$
\end{itemize}

\subsection{Política $\epsilon$-greedy}
Para a seleção de ações, adotou-se a política $\epsilon$-greedy. O agente escolhe a ação com maior valor $Q$ com probabilidade $1-\epsilon$ (explotação) e uma ação aleatória com probabilidade $\epsilon$ (exploração). Conforme especificado no enunciado, em caso de empate nos valores $Q$ durante a escolha gulosa ($A^*$), o desempate é feito de forma aleatória entre as melhores ações.

% ==============================
% 5. ANÁLISE E RESULTADOS
% ==============================
\section{Análise e Resultados}

\subsection{Variação do Parâmetro $\epsilon$}

Abaixo apresentamos os gráficos de recompensa acumulada por episódio e a média móvel (janela de 10 episódios) para diferentes valores de $\epsilon$. O número de episódios foi fixado em 300 para melhor visualização da estabilidade a longo prazo.

\begin{figure}[H]
    \centering
    \begin{subfigure}[b]{0.48\textwidth}
        \includegraphics[width=\textwidth]{plot_eps_0.01.png}
        \caption{$\epsilon = 0.01$ (Muita Explotação)}
    \end{subfigure}
    \hfill
    \begin{subfigure}[b]{0.48\textwidth}
        \includegraphics[width=\textwidth]{plot_eps_0.1.png}
        \caption{$\epsilon = 0.1$ (Padrão sugerido)}
    \end{subfigure}
    \caption{Comparação de convergência: Baixa exploração vs Padrão.}
\end{figure}

\begin{figure}[H]
    \centering
    \begin{subfigure}[b]{0.48\textwidth}
        \includegraphics[width=\textwidth]{plot_eps_0.3.png}
        \caption{$\epsilon = 0.3$}
    \end{subfigure}
    \hfill
    \begin{subfigure}[b]{0.48\textwidth}
        \includegraphics[width=\textwidth]{plot_eps_0.7.png}
        \caption{$\epsilon = 0.7$}
    \end{subfigure}
    \caption{Comparação de convergência com exploração moderada a alta.}
\end{figure}

\begin{figure}[H]
    \centering
    \begin{subfigure}[b]{0.48\textwidth}
        \includegraphics[width=\textwidth]{plot_eps_0.5.png}
        \caption{$\epsilon = 0.5$ (Equilibrado)}
    \end{subfigure}
    \hfill
    \begin{subfigure}[b]{0.48\textwidth}
        \includegraphics[width=\textwidth]{plot_eps_0.9.png}
        \caption{$\epsilon = 0.9$ (Muita Exploração)}
    \end{subfigure}
    \caption{Impacto de alta taxa de exploração na convergência.}
\end{figure}

Os experimentos foram realizados variando o parâmetro $\epsilon$ em $\{0.01, 0.1, 0.3, 0.5, 0.7, 0.9\}$. As médias totais de recompensa obtidas foram:

\begin{itemize}
    \item $\epsilon = 0.01$: Média de $15.11$ (Melhor desempenho)
    \item $\epsilon = 0.1$: Média de $-6.05$
    \item $\epsilon = 0.3$: Média de $-11.41$
    \item $\epsilon = 0.5$: Média de $-9.14$
    \item $\epsilon = 0.7$: Média de $-16.11$
    \item $\epsilon = 0.9$: Média de $-16.07$ (Pior desempenho)
\end{itemize}

Os dados mostram que valores muito baixos de exploração ($\epsilon=0.01$) resultaram nas melhores médias globais. Isso ocorre devido à natureza perigosa do ambiente: como as penalidades para erros são severas ($-20$ para o tóxico e $-5$ para a lama), qualquer movimento aleatório indesejado após o aprendizado do caminho ótimo reduz drasticamente a pontuação acumulada. Com $\epsilon \ge 0.5$, o agente colide frequentemente com obstáculos, explicando as médias negativas.

\subsection{Política Ótima Encontrada}

Abaixo apresentamos a tabela Q final e a política ótima derivada.

\begin{table}[H]
\centering
\footnotesize
\begin{tabular}{|c|r|r|r|r|}
\hline
\textbf{Estado} & \textbf{CIMA} & \textbf{DIREITA} & \textbf{BAIXO} & \textbf{ESQUERDA} \\ \hline
(1, 1) & 10.48 & -0.76 & $-\infty$ & $-\infty$ \\ \hline
(1, 2) & 12.41 & -4.00 & $-\infty$ & -0.60 \\ \hline
(1, 3) & 0.00 & 0.00 & $-\infty$ & 0.00 \\ \hline
(1, 4) & 8.43 & $-\infty$ & $-\infty$ & -4.00 \\ \hline
(2, 1) & -0.66 & 12.58 & -0.79 & $-\infty$ \\ \hline
(2, 2) & 1.83 & 14.29 & -0.36 & 1.86 \\ \hline
(2, 3) & 16.10 & -1.00 & -4.00 & 2.36 \\ \hline
(2, 4) & -0.36 & $-\infty$ & -0.93 & 14.29 \\ \hline
(3, 1) & -1.00 & -1.00 & 10.93 & $-\infty$ \\ \hline
(3, 2) & -4.00 & 16.06 & -0.20 & -0.24 \\ \hline
(3, 3) & 18.00 & 7.85 & 2.83 & 0.00 \\ \hline
(3, 4) & 19.85 & $-\infty$ & 1.10 & -0.20 \\ \hline
(4, 1) & $-\infty$ & -4.00 & 8.43 & $-\infty$ \\ \hline
(4, 2) & $-\infty$ & 0.00 & 0.00 & 0.00 \\ \hline
(4, 3) & $-\infty$ & 20.00 & 3.22 & 0.00 \\ \hline
(4, 4) & $-\infty$ & $-\infty$ & 0.00 & 0.00 \\ \hline
\end{tabular}
\caption{Valores Q finais aprendidos (Média de 300 episódios, $\epsilon=0.01$)}
\label{tab:qtable_numeric}
\end{table}

\begin{table}[H]
\centering
\begin{tabular}{|c|c|c|c|}
\hline
(4,1) $\downarrow$ BAIXO & (4,2) \textbf{TÓXICO} & (4,3) $\rightarrow$ DIREITA & (4,4) \textbf{SAÍDA} \\ \hline
(3,1) $\downarrow$ BAIXO & (3,2) \textbf{LAMA} & (3,3) $\rightarrow$ DIREITA & (3,4) $\uparrow$ CIMA \\ \hline
(2,1) $\rightarrow$ DIREITA & (2,2) $\rightarrow$ DIREITA & (2,3) $\uparrow$ CIMA & (2,4) \textbf{LAMA} \\ \hline
(1,1) $\uparrow$ CIMA & (1,2) $\uparrow$ CIMA & (1,3) \textbf{TÓXICO} & (1,4) $\uparrow$ CIMA \\ \hline
\end{tabular}
\caption{Política Ótima obtida com $\epsilon=0.01$}
\label{tab:policy_final}
\end{table}

% ==============================
% 6. CONCLUSÃO
% ==============================
\section{Conclusão}

O algoritmo Q-Learning foi capaz de convergir para uma solução ótima no ambiente do laboratório simulado. A introdução da fundamentação teórica de Bellman permitiu compreender como os valores explodiram em magnitude devido à formulação acumulativa, mas ainda assim preservaram a ordem de preferência correta para a navegação. 

A análise da variação do parâmetro $\epsilon$ evidenciou o dilema exploração-explotação: taxas muito altas de exploração impedem a estabilização da recompensa máxima, enquanto taxas muito baixas aceleram a convergência mas aumentam o risco de mínimos locais. Para este ambiente específico, um $\epsilon=0.01$ provou ser o mais eficiente.

\clearpage

\end{document}